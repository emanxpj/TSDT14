\documentclass[10pt]{article}

\usepackage{times}
\usepackage{mathptmx}
\usepackage{spverbatim}
\usepackage{amsmath}
\usepackage{mathtools}

\raggedbottom
\sloppy

\title{Time-discrete Stochastic Signals\\
\emph{TSDT14}}

\author{David Habrman \\ davha227, 920908-2412\\
Jens Edhammer \\ jened502, 920128-5112 }

\date{\today}

\begin{document}

\maketitle

\section{Introduction}
This is a laboration report in signal theory (TSDT14). It consists of three studies.

\subsection{Study 1 – Modelling Signals}
Study 1 involves analyzing the power spectral density (PSD) and auto correlation function (ACF)
of noise filtered through two diffrent filters. One close to ideal and one simple low pass filter.
The ACF and PSD were first theoretically calculated and then estimated.
The ACF was estimated using Blackman-Tukey's- and Barlett's -estimate and the
PDF was estimated using periodograms, averaged periodograms and smoothing.

\subsection{Notation}
The following notations will be used. \\
PSD - Power Spectral Density \\
ACF - Auto correlation function \\
$\hat{r}$ - Estimated function or value r. \\
WGN - White Gaussian Noise

\section{Method}
\subsection{Study 1 – Modelling Signals}

White Gaussian noise is used in this study. It was filtered through the two
 filters described in equation~\ref{eq:simpleH} and equation~\ref{eq:idealH}.
 The ideal filter is approximated by a highorder Butterworth lowpass filter.

\begin{equation}
  \label{eq:simpleH}
  H[\theta] =\frac{1-a}{1-ae^{-j2\pi\theta }}
\end{equation}

\begin{equation}
  \label{eq:idealH}
  H[\theta] =rect(\frac{\theta}{\theta_0} )
\end{equation}

\subsubsection{Theoretical ACF and PSD}
PSD of WGN is constant and our WGN has a variance of one. Due to this the PSD of our noise, $R_x$, is chosen to one.
The PSD of the filtered WGN is then calculated using equation~\ref{eq:PSDformula}.

\begin{equation}
  \label{eq:PSDformula}
  R_y[\theta] = R_x|H(\theta)|^2;
\end{equation}

Because the ACF is the inverse transformed PSD it can be calculated using equation~\ref{eq:ACFformula} where N is the number of samples.

\begin{equation}
  \label{eq:ACFformula}
  r_y[n] = \frac{1}{N}\sum_{k=0}^{N-1}R_ye^{j2\pi kn/N}
\end{equation}

\subsubsection{Estimated ACF and PSD}
The ACF was to be estimated using Blackman-Tukey's- and Barlett's -estimate. This was done using equation~\ref{eq:BmanT} and equation ~\ref{eq:Blett}.

\begin{equation}
\label{eq:BmanT}
\hat{r}_y[k] = \frac{1}{N-|k|}\sum_{n=0}^{N-|k|-1}r_x[n+|k|]r_x[n], |k|<N \\
		0 elsewhere
\end{equation}

\begin{equation}
\label{eq:Blett}
\hat{r}_y[k] = \frac{1}{N}\sum_{n=0}^{N-|k|-1}r_x[n+|k|]r_x[n], |k|<N \\
		0 elsewhere
\end{equation}



The PDF was in turn to be estimated using periodograms, averaged periodograms and smoothing. Periodograms was done using equation~\ref{eq:Peri}. Where $Y_N$, the filtered noise in the frequency domain, was calculated using equation~\ref{eq:YN}. Averaged periodograms was done by calculating a mean of the periodogram inside an interval and repeating this for all samples. The averaged periodogram was then set to the mean inside each  interval.

\begin{equation}
\label{eq:Peri}
\hat{R}_y[\theta] = \frac{1}{N}|Y_n[\theta]|^2
\end{equation}

\begin{equation}
\label{eq:YN}
Y_N[\theta] = \sum_{n=0}^{N-1}y[n]e^{-j2\pi\theta n}
\end{equation}

Smoothing was done by convolving the periodogram with a lowpassfilter to get rid of high frequencies and thereby get a smoother estimate of the PSD.

\section{Result and conclusion}
\subsection{Study 1 – Modelling Signals}

Noted in equation~\ref{eq:ACFsimple} is the functions for the ACF
for the simple filter and equation~\ref{eq:ACFideal} for the close to ideal filter.
Equation~\ref{eq:PSDsimple} does the same for the simple filter's PSD
and equation~\ref{eq:PSDideal}, the close to ideal filter's PSD.

\begin{equation}
  \label{eq:ACFsimple}
  r_{y,simple} = ryt1 = R_x\frac{(1-a)}{(1+a)}a^{|n|};  |a| < 1
\end{equation}

\begin{equation}
  \label{eq:PSDsimple}
  R_{y,simple} =  R_x|\frac{1-a}{1-ae^{-j2\pi\theta}}|^2;
\end{equation}

\begin{equation}
  \label{eq:ACFideal}
  r_{y,Ideal} = R_x\theta_{0}sinc(n\theta_0);
\end{equation}

\begin{equation}
  \label{eq:PSDideal}
  R_{y,Ideal} = R_xrect(\frac{\theta - k}{\theta_0})
\end{equation}
\clearpage

\end{document}
