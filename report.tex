\documentclass[10pt]{article}

\usepackage{times}
\usepackage{mathptmx}
\usepackage{amsmath}
\usepackage{mathtools}

\raggedbottom
\sloppy

\title{Time-discrete Stochastic Signals\\
\emph{TSDT14}}

\author{David Habrman \\ davha227, 920908-2412\\
Jens Edhammer \\ jened502, 920128-5112 }

\date{\today}

\begin{document}

\maketitle

\section{Introduction}
This is a laboration report in signal theory (TSDT14). It consists of three studies.

\subsection{Study 1 – Modelling Signals}
Study 1 involves analyzing the power spectral density (PSD) and auto correlation function (ACF)
of noise filtered through two diffrent filters. One close to ideal and one simple low pass filter.
The ACF and PSD were first theoretically calculated and then estimated.
The ACF was estimated using Blackman-Tukey's- and Barlett's -estimate and the
PDF was estimated using periodograms, averaged periodograms and smoothed periodograms.

\subsection{Notation}
The following notations will be used. \\
PSD - Power Spectral Density \\
ACF - Auto correlation function \\
$\hat{r_y}$ - Estimated ACF of signal y. \\
$\hat{R_y}$ - Estimated PSD of signal y. \\
WGN - White Gaussian Noise

\section{Method}
\subsection{Study 1 – Modelling Signals}

White Gaussian noise is used in this study. It was filtered through the two
 filters described in equation~\ref{eq:simpleH} and equation~\ref{eq:idealH},
  to produce our signal y.
 The ideal filter is approximated by a highorder Butterworth lowpass filter.

\begin{equation}
  \label{eq:simpleH}
  H[\theta]_{simple} =\frac{1-a}{1-ae^{-j2\pi\theta }}
\end{equation}

\begin{equation}
  \label{eq:idealH}
  H[\theta]_{ideal} =rect(\frac{\theta}{\theta_0} )
\end{equation}

\subsubsection{Theoretical ACF and PSD}
PSD of WGN is constant and our WGN has a variance of one. Due to this the PSD
of our noise, $R_x$, is chosen to one.
The PSD of the filtered WGN is then calculated using equation~\ref{eq:PSDformula}.

\begin{equation}
  \label{eq:PSDformula}
  R_y[\theta] = R_x|H(\theta)|^2;
\end{equation}

The ACF can be calculated using inverse fourier transform, see equation~\ref{eq:ACFformula}

\begin{equation}
  \label{eq:ACFformula}
  r_y[n] = \mathcal{F}^{-1}\{R_y\}[n]
\end{equation}

\subsubsection{Estimated ACF and PSD}
The ACF was to be estimated using Blackman-Tukey's- and Barlett's -estimate.
This was done using equation~\ref{eq:BmanT} and equation ~\ref{eq:Blett}.

\begin{equation}
\label{eq:BmanT}
\hat{r}_y[k] = \frac{1}{N-|k|}\sum_{n=0}^{N-|k|-1}r_x[n+|k|]r_x[n], |k|<N
		0 elsewhere
\end{equation}

\begin{equation}
\label{eq:Blett}
\hat{r}_y[k] = \frac{1}{N}\sum_{n=0}^{N-|k|-1}r_x[n+|k|]r_x[n], |k|<N
		0 elsewhere
\end{equation}

The PSD was estimated using periodograms, averaged periodograms
and smoothing. The raw periodogram was calculated by taking the fouriertransform
of the estimated ACF.
Averaged periodograms was done by calculating a mean of the periodogram inside
an interval and repeating this for all samples. The averaged periodogram was
then set to the mean inside each interval.

% \begin{equation}
% \label{eq:YN}
% Y_N[\theta] = \sum_{n=0}^{N-1}y[n]e^{-j2\pi\theta n}
% \end{equation}

The smoothed periodogram was done by first multiplying the estimated ACF with a
 suitable window and then using the fourier transform to get the estimated PSD,
 see equation~\ref{eq:win}. This can be viewed as a form of moving average in the fourier-domain.

 \begin{equation}
 \label{eq:win}
 \hat{R}_{y,smoothed} = \mathcal{F}\{\hat{r}_yw[k]\}
 \end{equation}


\section{Result and conclusion}
\subsection{Study 1 – Modelling Signals}

Noted in equation~\ref{eq:ACFsimple} is the theoretically calculated functions for the ACF
for the simple filter and equation~\ref{eq:ACFideal} for the close to ideal filter. The functions are illustrated in figure1 and figure2.
Equation~\ref{eq:PSDsimple} does the same for the simple filter's PSD
and equation~\ref{eq:PSDideal}, the close to ideal filter's PSD. These functions are illustrated in figure3 and figure4.

\begin{equation}
  \label{eq:ACFsimple}
  r_{y,simple} = R_x\frac{(1-a)}{(1+a)}a^{|n|};  |a| < 1
\end{equation}

\begin{equation}
  \label{eq:PSDsimple}
  R_{y,simple} =  R_x|\frac{1-a}{1-ae^{-j2\pi\theta}}|^2;
\end{equation}

\begin{equation}
  \label{eq:ACFideal}
  r_{y,ideal} = R_x\theta_{0}sinc(n\theta_0);
\end{equation}

\begin{equation}
  \label{eq:PSDideal}
  R_{y,ideal} = R_xrect(\frac{\theta - k}{\theta_0})
\end{equation}

Insert pictures of theoretical ACF:s and PDF:s.\\

ACF estimations, of the signal filtered through the simple filter, using Blackman-Turkey's and Barlett's-estimate are shown in figure5. The estimations of the signal filtered through the near ideal filter are shown in figure6.\\

Insert pictures of ACF estimations for simple and ideal filter.\\

Comment on estimations\\

PDF estimations of the signal filtered through the simple filter are shown in figure7 and figure8 while the estimations of the signal filtered through the near ideal filter are shown in figure9 adn figure10.\\

Insert pictures of PDF estimations for simple and ideal filter.\\

Comment on estimations\\


\end{document}
